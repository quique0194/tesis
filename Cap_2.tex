\chapter{Marco te�rico: Aprendizaje por refuerzo en sistemas multiagente}

Le permite a un agente escoger acciones intereactuando con el ambiente. Modela el mundo mediante un \ac{MDP}.


\section{Modelo b�sico}

Consiste en las siguientes partes:
\begin{itemize}
    \item Conjunto de estados $S$. Un estado es un conjunto de caracter�sticas que indican como est� el ambiente.
    \item Conjunto de acciones $A$. 
    \item Reglas de transici�n entre los estados
    \item Reglas que determinar la recompensa inmediata de una transici�n
    \item Reglas que describen lo que observa el agente
\end{itemize}


\subsection{Ecuaci�n de bellman}

Ecuaci�n de bellman


\subsubsection{M�todo de Montecarlo}

M�todo de Montecarlo


\subsubsection{M�todo de diferencias temporales}

M�todo de diferencias temporales


\subsection{Q-learning}

Q-learning


\subsection{Exploraci�n vs Explotaci�n}

Exploraci�n vs Explotaci�n


\subsection{Sistemas multiagente}

Definicion
Tipos
Equilibrio de Nash


\subsection{Aprendizaje en sistemas multiagente}

Aprendizaje como un equipo
Aprendizaje independiente
Aprendizaje de acciones conjuntas
Aprendizaje con valores de influencia


\section{Consideraciones Finales}

Cada cap�tulo excepto el primero debe contener al finalizarlo una secci�n de consideraciones que enlacen
el presente cap�tulo con el siguiente.
