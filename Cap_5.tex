\chapter{Propuesta}

Aquí se definiran los estados y las acciones. Se calculará el número de posibles estados y el número de posibles acciones. Se demostrará que ambos son virtualmente infinitos por lo que la creación de una tabla es inviable. Entonces quedará justificado la interpolación de la función Q con una \ac{DBN}.

Se describira la arquitectura de \ac{DBN} que se planea utilizar, así como la entrada y la salida de la red. También se describirán los parámetros del aprendizaje por refuerzo.

