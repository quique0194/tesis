\chapter{Propuesta}

Aplicar Deep Q Learning al aprendizaje de jugadas de futbol de bajo nivel.

\begin{itemize}
\item Se tiene un simulador hecho a la medida
\item Se necesita un sistema de scoring que le de recompensas al agente por pasos intermedios, como coger el balon, patearlo, perderlo, sacar el balon fuera del campo y hacer un gol.
\item Es necesario un aprendizaje icremental, donde la primera etapa, la jugaria con un adversario que no se mueve, luego con un adversario aleatorio y finalmente con una estrategia de arbol de decision
\item Tambien podemos escalar este aprendizaje en un 1 vs 1, 2 vs 2 y finalmente un 3 vs 3
\item Para tener una idea de que tan bien se desempeña el algoritmo, lo compararemos con el desempeño de un jugador humano
\item Se utilizaran 2 modelos de aprendizaje, el primero un deep q learning con una mlp por detras, y el segundo un deep q learning con una red convolucional por detras, que aprendera directamente de los frames generados por el juego
\end{itemize}
