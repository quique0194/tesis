\chapter{Introducción}

% Tengo 2 opciones:
% Usar la comparacion como centro del trabajo, pero no tiene mucho sentido, puesto que ya se sabe que dqn es mejor. No hay mucho q comparar
% Queda: Ejecutar una tecnica reciente en un problema importante y describir resultados, comparar con el estado del arte
%


\section{Motivación y Contexto}

La robocup tiene como objectivo construir un equipo robótico capaz de jugar fútbol con el campeón mundial y ganar en el año 2050. Con este fin se tienen varias categorias de competición que retan a los investigadores de todo el mundo para ir mejorando poco a poco los algoritmos y técnicas utilizadas en la construcción de robots.

Una de esas categorías es la simulación de fútbol 2D, que como su nombre lo indica es una simulación. La ventaja de una simulación es que abstrae todos los detalles de construcción del hardware de los robots y permite a los investigadores centrarse en los algoritmos, en la estrategia. Gracias a esto, esta categoría sirve como una cama de pruebas muy interesante para probar algoritmos de inteligencia artificial nuevos.

Recientemente se anunció la noticia de que finalmente se había creado una inteligencia artificial capaz de ganarle al campeón mundial de Go en una ronda de 5 partidas, donde quedaron 4 contra 1. El algoritmo que logró esta hazaña combina 2 grandes ramas de la inteligencia artificial: el aprendizaje por refuerzo y el aprendizaje profundo. Con este mismo algoritmo también se creó una inteligencia artificial capaz de aprender a jugar 49 juegos de Atari teniendo como entrada solamente los píxeles de la pantalla y la puntuación.

En este trabajo se pretende dar un paso inicial hacia la creación de un equipo robotico simulado, utilizando un algoritmo del estado del arte actual, como es el aprendizaje por refuerzo profundo.


\section{Planteamiento del Problema}

El fútbol de robots simulado es un problema muy difícil que se puede modelar con un \ac{MDP} con número de estados virtualmente infinito. El problema que se quiere abordar en este proyecto es el aprendizaje de jugadas individuales de un agente en la modalidad de aprendizaje por refuerzo.


\section{Objetivos}

Que un agente aprenda a esquivar rivales y anotar goles usando un algoritmo de aprendizaje por refuerzo profundo.


\subsection{Objetivos Específicos}

\begin{itemize}
\item Modelar el problema como un \ac{MDP} con sus respectivos estados, acciones y función de recompensa
\item Implementar el algoritmo \ac{DQN} en un agente y entrenarlo
\item Analizar el desempeño del agente en la tarea dada
\end{itemize}


\section{Organización de la tesis}

Al final...