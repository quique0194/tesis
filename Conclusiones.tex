\chapter{Conclusiones y Trabajos Futuros}\label{chap:conclusiones}

Se vio que los algoritmos de aprendizaje por refuerzo han recibido un impulso con la aparición de la Deep Q-Network, ya que nuevamente han llamado la atención de los investigadores y se están desarrollando varias líneas de investigación relacionadas con el tema. Una de esas líneas es la aplicación del aprendizaje por refuerzo en contextos con espacios de acciones continuas.

Con Deep Q-Network es posible calcular el valor de las acciones con una función paramétrica. Siendo las más comúnmente utilizadas las redes neuronales profundas, ya que permiten trabajar con grandes cantidades de estados continuos. Sin embargo, su aplicación no es trivial, ya que se generan inestailidades que se resuelven con la aplicación de dos ideas claves. La primera es usar una memoria de replay, y la segunda, actualizar los valores de las acciones periódicamente.

\section{Problemas encontrados}
Falta una mejor aplicación de las estrategias multiagente. Por ahora sólo se tiene un aprendizaje independiente.

\section{Recomendaciones}
Se recomienda utilizar un código base de alguno de los equipos ganadores de la categoría de simulación de la RoboCup de años pasados. Ya que la programación de un equipo básico, capaz de comunicarse y sincronizarse con el servidor, además de modelar el mundo, no es sencilla.

\section{Trabajos futuros}
Como trabajo futuro sería bueno implementar mejores estrategias de aprendizaje multiagente, como por ejemplo aprendizaje con valores de influencia.